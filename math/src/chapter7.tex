\section*{Chapter 7: Deriving a Generic Algorithm}

\paragraph{Exercise 7.1}
How many additions are needed to compute \texttt{fib0(n)}?

\begin{proof}[Answer]
Let $\alpha(n)$ be the number of additions needed to compute
\texttt{fib0(}$n$\texttt{)}. $\alpha(n)$ can be characterized by
the following recurrence relation:
$$
\alpha(n) = 
\begin{cases}
    0 & \textrm{if } n \leq 1 \\
    1 + \alpha(n-1) + \alpha(n-2) & \textrm{if } n \geq 2
\end{cases}
$$

It can be shown by induction on $n$ that $\alpha(n) = F_{n+1} - 1$.
In fact, if $n \leq 1$, $\alpha(n) = 0 = F_{n+1} - 1$, since
by definition $F_1 = F_2 = 1$. For $n \geq 2$,
\begin{eqnarray*}
\alpha(n)  &=& 1 + \alpha(n-1) + \alpha(n-2) \\
           &=& 1 + (F_n - 1) + (F_{n-1} - 1) \\
           &=& (F_n + F_{n-1}) - 1 \\
           &=& F_{n+1} - 1
\end{eqnarray*}
Thus, the number of additions we seek is
$\alpha(n) = F_{n+1} -1 \in \Theta(\varphi^n)$, where $\varphi$ is the 
golden ratio.
\end{proof}
