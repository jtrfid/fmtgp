\section*{Chapter 5: The Emergence of Modern Number Theory}

\paragraph{Exercise 5.1}
Prove that if $n > 4$ is composite, then $(n-1)!$ is a multiple of $n$.

\begin{proof}
Let $n > 4$ be a composite integer with prime factorization
$n = p_1^{\alpha_1} \dots p_k^{\alpha_k}$. First note that, if $d$ is a 
proper divisor of $n$, then $\Divides{d}{(n-1)!}$. Indeed, $d \leq n-1$,
and so $(n-1)! = (n-1) \cdot (n-2) \cdots d \cdot (d-1) \cdots 1$. \\

Suppose that $k > 1$. Then, $\Divides{p_i^{\alpha_i}}{(n-1)!}$,
$1 \leq i \leq k$. Since $p_1^{\alpha_1}, \dots, p_k^{\alpha_k}$ are
pairwise coprime, we have that
$\Divides{n = p_1^{\alpha_1} \dots p_k^{\alpha_k}}{(n-1)!}$.\\

Now, suppose that $k = 1$. Since
$n$ is composite and $n = p_1^{\alpha_1} > 4$, then either $p_1 > 2$ or
otherwise $\alpha_1 > 2$. In the latter case, 
$(n-1)! = (n-1) \cdot (n-2) \cdots p_1^{\alpha_1 - 1} \cdots p_1 \cdots 1$,
and so $\Divides{n = p_1^{\alpha_1}}{(n-1)!}$. Otherwise, if $\alpha_1 = 2$,
$2 \cdot p_1 < p_1^2 = n$, and so
$(n-1)! = (n-1) \cdot (n-2) \cdots 2p_1 \cdots p_1 \cdots 1$, which means 
that $\Divides{n = p_1^{2}}{(n-1)!}$.
\end{proof}
