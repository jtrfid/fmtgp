\section*{Chapter 4: Euclid's Algorithm}

\paragraph{Exercise 4.3}
Prove that $\KthRoot{3}{16} + \KthRoot{3}{54} = \KthRoot{3}{250}$

\begin{proof}
$16 = 2^4$, $54 = 2 \cdot 3^3$ and $250 = 2 \cdot 5^3$. Then,
\begin{eqnarray*}
\KthRoot{3}{16} + \KthRoot{3}{54} &=& \KthRoot{3}{2^4} +
                                      \KthRoot{3}{2 \cdot 3^3} \\
                                  &=& 2 \KthRoot{3}{2} + 3 \KthRoot{3}{2} \\
                                  &=& 5 \KthRoot{3}{2} \\
                                  &=& \KthRoot{3}{5^3} \cdot \KthRoot{3}{2} \\
                                  &=& \KthRoot{3}{2 \cdot 5^3} \\
                                  &=& \KthRoot{3}{250}
\end{eqnarray*}
\end{proof}

\paragraph{Exercise 4.4}
Prove that, for any odd square number $x$, there is an even square number $y$
such that $x+y$ is a square number.

\begin{proof}
Since $x$ is square and odd, there must be an $n \in \Nat$ such that
$x = (2n+1)^2$. Let $y$ be some even square number. Thus, there must be an
$m \in \Nat$ such that $y = (2m)^2$. It follows that
\begin{eqnarray*}
x+y &=& (2n+1)^2 + (2m)^2 \\
    &=& 4(n^2 + m^2) + 4n + 1 
\end{eqnarray*}
We must define $m$ as a function of $n$ in such a way that this number conforms
a square. In order to do this, let's see what happens for some small cases:
\begin{itemize}
    \item If $n = 1$, then $x = 9$. If we set $m = 2$, $y = 16$ and $x+y = 25$,
    which is a square.
    \item If $n = 2$, then $x = 25$. Taking $m = 6$, $y = 144$ and $x+y = 169$,
    which is a square (since $13^2 = 169$).
    \item If $n = 3$, then $x = 49$. Now, $m$ can be $12$, and then $y = 576$
    and $x+y = 625 = 25^2$.
\end{itemize}
A careful analysis of these cases reveals a pattern: $m = n^2 + n$. Substituting
this in the equation shown before,
\begin{eqnarray*}
x+y &=& 4(n^2 + m^2) + 4n + 1  \\
    &=& 4(n^2 + (n^2 + n)^2) + 4n + 1 \\
    &=& 4n^2 + 4(n^2 + n)^2 + 4n + 1 \\
    &=& 4(n^2 + n)^2 + 4(n^2 + n) + 1 \\
    &=& (2(n^2 + n) + 1)^2 \\
\end{eqnarray*}
\end{proof}


\newsavebox\sqpython
\begin{lrbox}{\sqpython}
    \begin{minipage}[t]{3in}
         \vspace{4px}
         \begin{verbatim}
def find_squares(x, y):
    x1, x2 = x
    y1, y2 = y
    n = (x1**2 + x2**2)*(y1**2 + y2**2)
    return [(i,j) for i in xrange(n)
                  for j in xrange(i,n)
                  if n == i**2 + j**2]
         \end{verbatim}
    \end{minipage}
\end{lrbox}

\paragraph{Exercise 4.5}
Prove that, if $x$ and $y$ are both sums of two squares, then so is their
product $xy$.

\begin{proof}
Being $x$ and $y$ both sums of two squares, we can write them like so:
\begin{eqnarray*}
    x &=& x_1^2 + x_2^2 \\
    y &=& y_1^2 + y_2^2
\end{eqnarray*}
This implies that
\begin{eqnarray*}
    xy &=& (x_1^2 + x_2^2) \, (y_1^2 + y_2^2)  \\
       &=& x_1^2 y_1^2 + x_1^2 y_2^2 + x_2^2 y_1^2 + x_2^2 y_2^2
\end{eqnarray*}
After several failed attempts at completing the squares (i.e., adding and
subtracting the same thing), the following Python
script was used to gain some insight into the underlying pattern of $xy$:
    \begin{center}
    \framebox[8.5cm]{
    \usebox\sqpython}
    \end{center}
For example, 
\begin{itemize}
    \item \texttt{find\_squares((2,3), (5,7))} $\rightarrow$ \texttt{[(1, 31), (11, 29)]}.
    \item \texttt{find\_squares((1,2), (3,4))} $\rightarrow$ \texttt{[(2, 11), (5, 10)]}.
\end{itemize}
Playing with this script and guessing how to combine the elements in the input
tuples in order to generate an output tuple $(z_1, z_2)$, the following
pattern emerged: 
\begin{eqnarray*}
    z_1 &=& x_2 y_1 + x_1 y_2  \\
    z_2 &=& x_2 y_2 - x_1 y_1
\end{eqnarray*}
Indeed,
\begin{eqnarray*}
    z_1^2 + z_2^2 &=& (x_2 y_1 + x_1 y_2)^2 + (x_2 y_2 - x_1 y_1)^2  \\
                  &=& ((x_2 y_1)^2 + (x_1 y_2)^2 + 2 x_2 y_1 x_1 y_2)
                      + ((x_2 y_2)^2 + (x_1 y_1)^2 - 2 x_2 y_2 x_1 y_1) \\
                  &=& x_2^2 y_1^2 + x_1^2 y_2^2 + x_2^2 y_2^2 + x_1^2 y_1^2 \\
                  &=& x_1^2 y_1^2 + x_1^2 y_2^2 + x_2^2 y_1^2 + x_2^2 y_2^2 \\
                  &=& xy
\end{eqnarray*}

\end{proof}
